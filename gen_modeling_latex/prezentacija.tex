%----------------------------------------------------------------------------------------
%    PACKAGES AND THEMES
%----------------------------------------------------------------------------------------

\documentclass[aspectratio=169, xcolor=dvipsnames]{beamer}
\usetheme{SimpleDarkBlue}

\usepackage[sfdefault]{roboto}
\usepackage{graphicx}
\usepackage{booktabs}
\usepackage{hyperref}
\usepackage{float}
\usepackage{amsmath}
\usepackage{bm}

\setbeamerfont{normal text}{size=\large}
\setbeamerfont{frametitle}{size=\Large, series=\bfseries}
\setbeamerfont{title}{size=\huge, series=\bfseries}

%----------------------------------------------------------------------------------------
%    TITLE PAGE
%----------------------------------------------------------------------------------------

\title{Generativno modeliranje u računalnom vidu}
\subtitle{Seminar}

\author{Dominik Barukčić}

\institute{
	Fakultet elektrotehnike i računarstva \\
	Sveučilište u Zagrebu
}

\date{\today}

\begin{document}
	
	\begin{frame}
		\titlepage
	\end{frame}
	
	\begin{frame}{Sadržaj}
		\tableofcontents
	\end{frame}
	
	\section{Svrha rada i kontekst}
	
	\begin{frame}{Svrha i korisnost rada}
		\begin{itemize}
			\item Cilj: razumjeti generativno modeliranje u računalnom vidu
			\item Široka primjena: medicina, umjetnost, sigurnost, itd.
			\item Fokus: GAN, VAE, difuzijski modeli + praktični eksperimenti
		\end{itemize}
	\end{frame}
	
	\begin{frame}{Pozadina i tehnološki kontekst}
		\begin{itemize}
			\item Generativni modeli: stvaraju podatke slične stvarnima
			\item Usporedba: generativni vs. diskriminativni pristup
			\item Brz razvoj dubokog učenja omogućio nove primjene
		\end{itemize}
	\end{frame}
	
	\section{Problem i rješenja}
	
	\begin{frame}{Problem i izazovi}
		\begin{itemize}
			\item Nedostatak kvalitetnih podataka za treniranje
			\item Visoki troškovi prikupljanja i etička ograničenja
			\item Potreba za generiranjem realističnih i korisnih slika
		\end{itemize}
	\end{frame}
	
	\begin{frame}{Pregled postojećih metoda}
		\begin{itemize}
			\item \textbf{GAN} – realistične slike, ali osjetljivi na treniranje
			\item \textbf{VAE} – stabilniji, ali manja vizualna kvaliteta
			\item \textbf{Difuzijski modeli} – visoka kvaliteta, ali spori
		\end{itemize}
	\end{frame}
	
	\begin{frame}{Vlastiti pristup}
		\begin{itemize}
			\item Fokus na analizu pretreniranih modela
			\item Nema potrebe za dugotrajnim treniranjem
			\item Eksperimenti na stvarnim primjerima
		\end{itemize}
	\end{frame}
	
	\section{Eksperimenti}
	
	\begin{frame}{Eksperiment 1: StyleGAN3 – generiranje lica}
		\begin{itemize}
			\item Korišten pretrenirani model StyleGAN3-R (FFHQ skup)
			\item Nasumično generirane realistične slike lica
			\item Primjena: umjetnost, privatnost, testiranje
		\end{itemize}
	\end{frame}
	
	\begin{frame}{Eksperiment 2: Real-ESRGAN – povećanje rezolucije}
		\begin{itemize}
			\item Ulaz: slika niske rezolucije (stari Zagreb)
			\item Izlaz: poboljšana slika s više detalja
			\item Primjena: restauracija, analiza starih snimki
		\end{itemize}
	\end{frame}
	
	\begin{frame}{Eksperiment 3: CycleGAN – prijenos stila}
		\begin{itemize}
			\item Ulaz: slika u Monetovom stilu
			\item Izlaz: realistični pejzaži s istom kompozicijom
			\item Izazovi: kvaliteta transformacije ovisi o slici
		\end{itemize}
	\end{frame}
	
	\begin{frame}{Eksperiment 4: VAE – uklanjanje šuma}
		\begin{itemize}
			\item Ulaz: MNIST slike s Gaussovim šumom
			\item Izlaz: rekonstruirane čiste slike znamenki
			\item Koristi se VAE kao denoising autoencoder
		\end{itemize}
	\end{frame}
	
	\section{Rezultati i zaključci}
	
	\begin{frame}{Postignuća i ograničenja}
		\begin{itemize}
			\item Postignuto: uspješna primjena 4 modela na zadatke
			\item Prednosti: jednostavna primjena, realistični rezultati
			\item Ograničenja: kvaliteta ovisi o ulazu i modelu
		\end{itemize}
	\end{frame}
	
	\begin{frame}{Budući razvoj}
		\begin{itemize}
			\item Hibridni modeli: spajanje VAE-GAN, difuzija + stil
			\item Bolja kontrola nad generiranim podacima
			\item Primjena u industriji, medicini, edukaciji
		\end{itemize}
	\end{frame}
	
	\begin{frame}{Zaključak}
		\begin{itemize}
			\item Generativno modeliranje nudi moćne alate za računalni vid
			\item GAN i VAE modeli su posebno korisni u primjeni
			\item Predstavljeni rezultati pokazuju praktičnu vrijednost
		\end{itemize}
	\end{frame}
	
	\begin{frame}
		\Huge{\centerline{\textbf{Hvala na pozornosti!}}}
		\Large{\centerline{Imate li pitanja?}}
	\end{frame}
	
\end{document}